\section{Sound Field Representations}

\subsection{Introduction}

In acoustics, the \textbf{sound field} can be represented as superposition of basis solution of Helmotz equations, such as plane waves, spherical waves or cylindrical waves.
Each type of basis soluiton is characterized by a specific set of indipendente variables (propagation direction, radial order, ...) which uniqueky define the waveform.
The general solution to the wave equation is formed by adding up these basic solution $\phi_n$, each weighted by a specific coefficient $a_n$ (expansion coefficient:

\[
\Psi(\underline{\mathrm{x}}, t) = \sum_{n} a_n \phi_n(\underline{\mathrm{x}}, t)
\]

These coefficients define the strength and influence of each wave component, allowing us to describe the complete sound field using a combination of simpler waves.

A \textbf{source-free volume} is a region of space devoid of active sound sources. These volumes can be classified into three main categories:
\[
\Psi(\mathbf{x},t) = \sum_{n} a_{n}\,\phi_{n}(\mathbf{x},t) 
\quad \text{or} \quad 
\Psi(\mathbf{x},\omega) = \int A(\boldsymbol{\xi},\omega)\,\phi(\mathbf{x};\boldsymbol{\xi},\omega)\, d\boldsymbol{\xi},
\]

\begin{itemize}
    \item \textbf{Internal Field:} All sources are located outside the region of interest;
    \item \textbf{External Field:} Sources are located within the region, but their influence is analysed externally;
    \item \textbf{Mixture:} For instance, a spherical shell configuration, where some sources are enclosed within a smaller radius and others are situated beyond a larger radius.
\end{itemize}

\begin{figure}[H]
    \centering
    \includegraphics[width=0.5\linewidth]{int ext sound fields.png}
    \caption{Internal and external sound field}
\end{figure}


\subsection{Plane Wave Representations}

Plane waves representation consists in defining a sound field as sum of weighted plane waves.
Two primary types of plane wave representations are considered in this section.

\textbf{Weyl representation}, derived from the inhomogeneous wave equation, includes both propagating and evanescent waves (not of interest here).
\textbf{Whittaker representation}, derived from the homogeneous wave equation, considers only propagating plane waves.
The Whittaker plane wave expansion is a general rapresentation of all \textbf{solution of the homogeneous Helmholtz equation}, which is expressed in cartesian coordinates as:

\[
p(\underline{r}, \omega) = e^{j \langle \underline{k}, \underline{r} \rangle}, \qquad \underline{k} \in \mathbb{R}^3
\]

The \textbf{synthesis} of the \textbf{sound field} can be expressed as a multidimensional inverse Fourier transform over the domain $\mathcal{D}$:

\[
p(\underline{r}, \omega) = \left( \frac{1}{2\pi} \right)^3 \int_{\mathcal{D}} A(\underline{k}) e^{j \langle \underline{k}, \underline{r} \rangle} \  d^3 \underline{k}, \qquad \mathcal{D} = \left\{ \underline{k} \in \mathbb{R}^3 : \| \underline{k} \| = \frac{\omega}{c} \right\}
\]
where A(k) encodes the amplitude and phase of each plane wave contribution to the integral.
An alternative expression in spherical coordinates - over a spherical surface $\mathcal{S}$ - is given by:

\[
p(\underline{r}, \omega) = \left( \frac{1}{2\pi} \right)^3 \int_{\mathcal{S}} A(\theta, \phi, \omega) e^{j \frac{\omega}{c} \langle \underline{\hat{k}}, \underline{r} \rangle} \sin(\theta) \  d\theta \  d\phi, \qquad \mathcal{S} = \left\{ \theta \in [0, \pi], \phi \in [0, 2\pi] \right\}
\]

The \textit{unit vector} \( \underline{\hat{k}} \) is expressed as:

\[
\underline{\hat{k}} = \begin{pmatrix}
    \sin(\theta) \cos(\phi) \\
    \sin(\theta) \sin(\phi) \\
    \cos(\theta)
\end{pmatrix}
\]

The function \( A(\theta, \phi, \omega) \) encodes amplitude and phase of each plane wave and is known as the \textbf{Herglotz density}, which is independent of the observation point \( \underline{r} \).

\subsection{Spherical Harmonics Expansion}

A \textbf{basis solution} to the \textbf{Helmholtz equation} in spherical coordinates can be expressed as:

\[ p(\underline{r}, \omega) = R(r) \Theta(\theta) \Phi(\phi) = R(r) Y_l^m(\theta, \phi) \]

The radial component \( R(r) \) can be expressed in two forms:

\[ R(r) = R_1 j_l \left( \frac{\omega}{c} r \right) + R_2 y_l \left( \frac{\omega}{c} r \right) \qquad \text{or} \qquad  R(r) = R_3 h_l^{(1)} \left( \frac{\omega}{c} r \right) + R_4 h_l^{(2)} \left( \frac{\omega}{c} r \right) \]

In the previous expressions, \( j_l(z) \) and \( y_l(z) \) are \textbf{spherical Bessel functions} while \( h_l^{(1)}(z) \) and \( h_l^{(2)}(z) \) are \textbf{spherical Hankel functions} - defined as:

\[ h_l^{(1)}(z) = j_l(z) + j y_l(z), \qquad h_l^{(2)}(z) = j_l(z) - j y_l(z) \]

The following plots are the result of the code script \verb|radialDependency.m|:

\begin{figure}[H]
    \centering
    \includegraphics[width=0.65\linewidth]{sph bessel.png}
    \caption{Spherical Bessel functions - first kind (left) and second kind (right)}
\end{figure}

The following plots are the result of the code script \verb|radialDependency.m|:

\begin{figure}[H]
    \centering
    \includegraphics[width=0.75\linewidth]{Hankel.png}
    \caption{Spherical Hankel functions - real and imaginary parts}
\end{figure}

The Helmholtz equation solution is expressed by means of \textbf{spherical harmonics} $R(r) Y_l^m(\theta, \phi)$:

\begin{figure}[H]
    \centering
    \includegraphics[width=0.4\linewidth]{shperical harmonics.png}
    \caption{Spherical harmonics up to order 3}
\end{figure}


The general \textbf{sound field} can be represented as the \textbf{superposition of spherical harmonic waves}:

\[ p(\underline{r}, \omega) = \sum_{l=0}^{\infty} \sum_{m=-l}^{l} \left( A_{lm}(\omega) h_l^{(1)} \left( \frac{\omega}{c} r \right) + B_{lm}(\omega) h_l^{(2)} \left( \frac{\omega}{c} r \right) \right) Y_l^m(\theta, \phi) \]

Alternatively:

\[ p(\underline{r}, \omega) = \sum_{l=0}^{\infty} \sum_{m=-l}^{l} \left( C_{lm}(\omega) j_l \left( \frac{\omega}{c} r \right) + D_{lm}(\omega) y_l \left( \frac{\omega}{c} r \right) \right) Y_l^m(\theta, \phi) \]

The sound field is completely characterized by the sets of coefficients \( A_{lm}, B_{lm} \) or by the sets of coefficients \( C_{lm}, D_{lm} \).

The choice of coefficients \( A_{lm} \), \( B_{lm} \), \( C_{lm} \) and \( D_{lm} \) is done in order to keep or discard solutions and depends on the location of the sound source and the type of wave behaviour under analysis.
This determines whether we use \textit{incoming} or \textit{outgoing} wave solutions in the spherical harmonic expansion.

The \textbf{spherical Bessel functions of the first kind} \( j_l\left(z\right) \) and $y_l\left(z\right)$ represent incoming acoustic fields generated by sources far away and observed within the interior region (interior problem).

On the other hand, the \textbf{spherical Hankel functions} \( h_l^{(1)}\left(z\right) \) and \( h_l^{(2)}\left(z\right) \) describe outgoing acoustic field generated by sources close to the origin, observed in the exterior region (exterior problem).
As \( z \to 0 \), the spherical Hankel function of the second kind diverges $\left(\abs{h_l^{(2)}(z)} \to \infty\right)$.
Therefore, \( h_l^{(2)}(z) \) is unsuitable for representing interior fields near the origin and is primarily used for exterior problems.

\subsubsection{Internal and External Sound Fields}

The \textbf{internal sound field} can be expressed using the inverse Spherical Harmonics expansion as:

\[
p(\underline{r}, \omega) = \sum_{l=0}^{\infty} \sum_{m=-l}^{l} C_{lm}(\omega) j_l\left(\frac{\omega}{c} r\right) Y_l^m(\theta, \phi) 
\]

The Spherical Harmonics expansion (analysis) for the internal sound field is given by:

\[
C_{lm}(\omega) = \frac{1}{j_l\left(\frac{\omega}{c} r\right)} \int_{0}^{2\pi} \int_{0}^{\pi} p(\underline{r}, \omega) Y_l^{-m}(\theta, \phi) \sin(\theta) \  d\theta \  d\phi 
\]

For the \textbf{external sound field}, the inverse Spherical Harmonics expansion is defined as:

\[
p(\underline{r}, \omega) = \sum_{l=0}^{\infty} \sum_{m=-l}^{l} B_{lm}(\omega) h_l^{(2)}\left(\frac{\omega}{c} r\right) Y_l^m(\theta, \phi)
\]

The corresponding Spherical Harmonics expansion (analysis) for the external sound field is:

\[
B_{lm}(\omega) = \frac{1}{h_l^{(2)}\left(\frac{\omega}{c} r\right)} \int_{0}^{2\pi} \int_{0}^{\pi} p(\underline{r}, \omega) Y_l^{-m}(\theta, \phi) \sin(\theta) \  d\theta \  d\phi
\]

\subsubsection{Bandlimited Spherical Harmonics Expansion}

The inverse spherical harmonics expansion can be limited to a finite order \( L - 1 \), resulting in the \textbf{bandlimited expansion}:

\[
p(\underline{r}, \omega) \approx \sum_{l=0}^{L-1} \sum_{m=-l}^{l} B_{lm}(\omega) \  h_{l}^{(2)}\left(\frac{\omega}{c} r\right) \  Y_{l}^{m}(\theta, \phi)
\]

This approximation effectively represents the original sound field using \( L^2 \) coefficients. The accuracy of this representation depends on the properties of the spherical Bessel functions \( j_l \), which exhibit distinct behaviours based on their order. 
Lower orders are more effective in describing the sound field close to the origin, while higher orders capture the field at greater distances. As the order \( l \) increases, the argument \( \left(\frac{\omega}{c}\right) r \) at which the maximum occurs also shifts outward.
A \textit{practical guideline} for determining the validity of the bandlimited expansion is given by:

\[
\frac{\omega \cdot  r_{L-1}}{c} < L - 1
\]

We consider \( r_{L-1} \) as the maximum radius where the expansion remains accurate.
The validity is inversely proportional to the temporal frequency \( \omega \), showing that \textbf{higher frequencies} require \textbf{more coefficients} to maintain precision.

\begin{figure}[H]
    \centering
    \includegraphics[width=0.24\linewidth]{a.png}
    \includegraphics[width=0.22\linewidth]{b.png}
    \includegraphics[width=0.23\linewidth]{c.png}
    \caption{Effects of bandlimited expansion - ideal field (left) and bandlimited fields with $L=25$ (center) and $L=12$ (right)}
\end{figure}

\subsection{Cylindrical Harmonics Expansion}

The cylindrical harmonics expansion follows a similar logic, applied to cylindrical coordinates.
The \textbf{internal sound field} can be expressed as:

\[
p(\underline{r}, \omega) = \sum_{m=-\infty}^{\infty} \left( \frac{1}{2\pi} \int_{-\infty}^{\infty} C_m(\omega, k_z) \  J_m(k_{\rho} \rho) \  e^{j k_z z} \  dk_z \right) e^{j m \phi}
\]

This formulation is valid in a cylindrical region around the \( z \)-axis, with a radius \( \rho \) smaller than the distance to the nearest source.
The expansion leverages the \textbf{Bessel functions} of the first kind \( J_m(k_{\rho} \rho) \), which describe the radial component in cylindrical coordinates.

\clearpage
